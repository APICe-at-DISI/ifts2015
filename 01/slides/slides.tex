%%%%%%%%%%%%%%%%%%%%%%%%%%%%%%%%%%%%%%%%%%%%%%%%%%%%%%%%%%%%%%%%%%%%%%%%%%%%%%%%
\documentclass[presentation]{beamer} %\mode<presentation>{\usetheme{sapere}}
\usetheme{CambridgeUS}
\usecolortheme{orchid}

\definecolor{themeColor}{HTML}{11AD1E}

\setbeamercolor*{structure}{bg=black,fg=themeColor}

\setbeamercolor*{palette primary}{use=structure,fg=white,bg=structure.fg}
\setbeamercolor*{palette secondary}{use=structure,fg=white,bg=structure.fg!75}
\setbeamercolor*{palette tertiary}{use=structure,fg=white,bg=structure.fg!50!black}
\setbeamercolor*{palette quaternary}{fg=white,bg=black}

\setbeamercolor{section in toc}{fg=black,bg=white}
\setbeamercolor{alerted text}{use=structure,fg=structure.fg!50!black!80!black}

\setbeamercolor{titlelike}{parent=palette primary,fg=structure.fg!50!black}
\setbeamercolor{frametitle}{bg=structure.fg!10!white,fg=structure.fg!50!black!80!black}

\setbeamercolor*{titlelike}{parent=palette primary}

\usepackage[utf8]{inputenc}
\usepackage{amssymb}
\usepackage{graphicx}
\usepackage{subfigure}
\usepackage{multirow}
\usepackage{hhline}
\usepackage{amsfonts,amstext,amssymb,wasysym}
\usepackage{fancyvrb}
\usepackage{alltt}
\usepackage{textcomp}
\usepackage{url}
\usepackage{multimedia,pgf}
\usepackage{geometry}
\usepackage{listings}
\usepackage{bibentry}
\usepackage{framed}
\usepackage{cleveref}
\nobibliography*

% Code highlighting
\definecolor{Fuchsia}{HTML}{8C368C}
\definecolor{OliveGreen}{HTML}{3C8031}

\newcommand{\il}[1]{{\it \textcolor{gray}{// #1}}} % inline comment
\newcommand{\km}[1]{\textcolor{purple}{#1}} % key mechanism primitives
\newcommand{\ex}[1]{\textcolor{blue}{#1}} % external imported Java values
\newcommand{\fc}[1]{\textcolor{Fuchsia}{#1}} % field calculus calls
\newcommand{\fn}[1]{\textcolor{blue}{#1}} % building block / function calls
\newcommand{\vb}[1]{\textcolor{OliveGreen}{#1}} % variables
\newcommand{\str}[1]{\textcolor{darkgray}{#1}} % strings

\newcommand{\bral}{\textrm{{\tt {\char '173}}}\,}
\newcommand{\brar}{\textrm{{\tt {\char '175}}}}
\newcommand{\var}{\texttt{x}}
\newcommand{\asgK}{~\texttt{=}~}
\newcommand{\letK}{\texttt{let}~}
\newcommand{\tupK}[1]{\texttt{[}#1\texttt{]}}
\newcommand{\lambdaK}[2]{\texttt{(}#1\texttt{)->}#2}
\newcommand{\bodyK}[1]{\bral\! #1\!\brar}
\newcommand{\dotK}{\texttt{.}}
\newcommand{\applyK}{\texttt{apply}}
\newcommand{\mname}{\ex{\texttt{m}}}
\newcommand{\aname}{\ex{\texttt{\#a}}}
\newcommand{\repK}[3]{\texttt{\fc{rep}(#1<-#2)}#3}
\newcommand{\ifK}[3]{\texttt{\fc{if}}(#1)#2\,\texttt{\fc{else}}\,#3}
\newcommand{\muxK}[3]{\texttt{\fc{mux}}(#1)#2\,\texttt{\fc{else}}\,#3}
\newcommand{\nbrK}[1]{\texttt{\fc{nbr}}#1}


\title[Introduction]{Elementi di Programmazione e Sviluppo di Applicazioni}

\author[Pianini, Viroli]{
Danilo Pianini e Mirko Viroli\\
\texttt{{\footnotesize \{danilo.pianini, mirko.viroli\}@unibo.it}}}


\institute[UniBo]
{\textsc{Alma Mater Studiorum}---Universit\`a di Bologna a Cesena}

\date[\today]{Corso di Istruzione e Formazione Tecnica Superiore\\
\scriptsize 2016-01-12 - Cesena, Italia
}

\pgfdeclareimage[height=0.625cm]{university-logo}{images/logo}
\logo{\pgfuseimage{university-logo}}


\begin{document}

\AtBeginSubsection[]{%
  \begin{frame}<beamer>
    \frametitle{Outline}
    \tableofcontents[currentsection,currentsubsection]
  \end{frame}
  \addtocounter{framenumber}{-1}% If you don't want them to affect the slide number
}

%===============================================================================
\frame[label=coverpage]{\titlepage}
%===============================================================================

\section*{Outline}
%===============================================================================
\frame{\tableofcontents}

%===============================================================================
\section{Contenuti del corso}
%===============================================================================

\subsection{Elementi di base del linguaggio C}

\begin{frame}
	\begin{block}{Contenuti}
		\begin{itemize}
			\item Compilazione ed esecuzione di programmi C
			\item Tipi di dato
			\item Istruzioni di controllo
			\item Ricorsività e iteratività
			\item ``Avanzate'': puntatori, array e strutture dati
			\item Input / Output
		\end{itemize}
	\end{block}
	\begin{block}{Obiettivi}
		\begin{itemize}
			\item Essere in grado di scrivere semplici programmi C
			\item Essere in grado di leggere programmi C fatti da altri
			\item Essere in grado di compilare ed eseguire programmi C
		\end{itemize}
	\end{block}
\end{frame}


\subsection{La programmazione orientata agli oggetti e il linguaggio Java}

\begin{frame}
	\begin{block}{Contenuti}
		\begin{itemize}
			\item Elementi di base di programmazione OO
			\item Incapsulamento e interfacce
			\item Ereditarietà e polimorfismo
			\item Eccezioni e meccanismi avanzati
			\item Input / Output
			\item Interfacce grafiche
		\end{itemize}
	\end{block}
	\begin{block}{Obiettivi}
		\begin{itemize}
			\item Essere in grado di compilare ed eseguire programmi Java
			\item Essere in grado di leggere e scrivere programmi Java
			\item Avere i mezzi per costruire applicazioni anche elaborate
		\end{itemize}
	\end{block}
\end{frame}

\subsection{Elementi di base di ingegneria del software}

\begin{frame}
	\begin{block}{Contenuti}
		\begin{itemize}
			\item Metodologie di sviluppo
			\item Modellare con UML
			\item Design patterns
			\item Debugging
			\item Testing
		\end{itemize}
	\end{block}
	\begin{block}{Obiettivi}
		\begin{itemize}
			\item Comprendere le fasi di sviluppo di un progetto software
			\item Capire le basi del linguaggio grafico UML
			\item Capire il concetto di design pattern e alcuni esempi
			\item Capire l'importanza delle tecniche di testing e debug, e saperne applicare alcune
		\end{itemize}
	\end{block}
\end{frame}

\section{Metodologia}

\begin{frame}{Approccio al corso}
	\begin{block}{Grado di approfondimento}
		\begin{itemize}
			\item Il programma è \textbf{molto} vasto, per essere trattato in modo molto approfondito richiede (almeno) tre corposi corsi universitari
			\item Il nostri obiettivi sono:
				\begin{itemize}
					\item Fornirvi una conoscenza degli argomenti suddetti sufficiente a rendervi \textbf{operativi}
					\item Farvi capire le basi in modo tale da rendervi capaci di approfondire anche in modo autonomo
				\end{itemize}
		\end{itemize}
	\end{block}
	\begin{block}{Flessibilità}
		\begin{itemize}
			\item La quantità di contenuti che riusciremo a sviscerare dipenderà da quanto bene e velocemente riuscirete a farli vostri
			\item Cercheremo di stimolarvi sugli argomenti che ci sembrano interessarvi maggiormente
		\end{itemize}
	\end{block}
\end{frame}

\begin{frame}{A lezione}
	\begin{block}{Molta pratica}
		Le lezioni saranno strutturate come spiegazioni seguite da esercizi.
		
		Il principio è: impara una cosa e mettila subito in pratica!
	\end{block}
	\begin{block}{Gli appunti sono importanti}
		Per alcuni argomenti (sicuramente per il primo modulo) spiegheremo di mano in mano senza slides.
		\begin{itemize}
			\item Interazione diretta con voi
			\item Superamento di ostacoli di comprensione man mano che si presentano
			\item Più flessibilità!
		\end{itemize}
		È importante che prendiate delle note!
	\end{block}
\end{frame}

\section*{\refname}
%===============================================================================

\end{document}

\begin{frame}[allowframebreaks]
%\begin{frame}[t,allowframebreaks]
  \frametitle{\refname}
  \scriptsize
  \bibliographystyle{alpha}
  \bibliography{bibliography}
\end{frame}
\section*{\refname}
